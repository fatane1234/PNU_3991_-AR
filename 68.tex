\documentclass[10pt,a4paper]{book}

\begin{document}

\begin{flushleft}
  \textbf{\!\!\!\!\!\!\!\!\!\!\!\!\!\!\!\!\!\!\!\!68} \quad CHAPTER FIVE
\end{flushleft}

Finally,it needs to be noted that there are certain circumstances when the Net should probably not be used to obtain consent.These circumstanes include situations where the consent of guardians of minors or that of persones of diminished mental capasity is required.We advise against this for the protection of \emph{both} the e-researcher and the participant.

\begin{flushleft}
  \textbf{\!\!\!\!\!\!\!\!\!\!\!\!\!\!\!\!\!\!\!\!WHEN IS CONSENT NEEDED?THE PUBLIC VERSUS}

  \textbf{\!\!\!\!\!\!\!\!\!\!\!\!\!\!\!\!\!\!\!\!PRIVATE DILEMMA}
\end{flushleft}

\!\!\!\!\!\!\!\!\!\!Informed consent is needed in almost all types of research with two notable exceptions.First,consent is generally not required to study an activity that is nonintrusive and takes place in a public space.For example,it is not necessary to obtain consent when undertaking naturalistic observations(for example,when studying linguistic patterns of fans' chanting at a football game).Neither is it usually neccessary to obtain permission when studying a public record or archive.For example,it is not necessary to obtain permission to study the public speeches of politicians,perform content analysis of newspapers or other mass media,or to study the public record of proceedings form a legislature.It is possible to argus that this notion of public space is appropriately extended to the study of activity on public newsgroups,mail lists,char rooms,or virtual reality environments(e.g.,MOOs,MUDs).Specifically,as these kinds of online spaces are open for anyone to join and,hence,can be interpreted as a public spaces,informed consent from every participant is not required since the researcher is often not participating and,thus,not affectain the interaction that takes place.

Or is it a public space? This interpretation is not as straightforward as some e-researchers would like it to be,as the sense of what is public or private is defined not by the technology,but by the perception of privacy and inclusion that is maintained by the participants.Imagine,for a moment,that you are in a public park and you need to use the public washroom.As you are leaving this public facility you notice there is a video camera in each of the washroom cubicles.How do you think you might feel to learn that this is part of a research project? As King(1996)notes,with this kind of example "The sense of violation possible is proportional to the expectation of privacy that group members had prior to learning they were studied." Forexample,studies with virtual self-help groups have shown remarkable candor among participants and the publication of this content has been viewed by some participants as a violation of the privacy of the group(Sharf,1999).An additional factor determining private space is the degree of intimacy that the researchers is studying.King notes that,generally,activity in a public place does not require informed consent.For example,noting how people are sitting on park benches.Nevertheless as Waskul and Douglass(1996)point out,if one installs a tape recorder and records conversations that take place on that park bench,a much different level of consent is required for ethical research.

Waskul and Douglass(1996)remind us further that "ethical consideration" should entail an interplay between codes of conduct and an intimate understanding of the nature of the online environment." To behave ethically requires explicit and expert knowledge of the context within which the researcher functions.The Net is made up

\end{document}
