\documentclass[10pt,a4paper]{book}

\begin{document}

\begin{flushright}
  \textmd{ETHICS AND THE E-RESEARCHER} \quad \textbf{69}
\end{flushright}

\!\!\!\!\!\!\!\!\!\!of a diverse set of technological and cultural contexts.For example,the ethics of analyzing the interaction in a large public discussion board sponsored by a media outlet such as the \emph{New York Times},call for far different means to protect privacy than research involving privante emails.Funther,codes of conduct may apply differentially to different types of research.For example,the study of anonymous language use in public online chat rooms and the publication of results requires different levels of individual disclosure than a study that is focused on identifying appropriate teacher/student interventions during an instructional class using the same Net-based chat technology.Thus,even though the technology is the same,different standards of ethical research behavior are required for these different research investigations.

To help the e-researcher determine when and what type of consent is required,many of the formal professional and research granting bodies provide guidelines that can help address some of the gray areas of ethical research.The 1994Canadian code of Ethical Conduct for Research Involving Humans( http:// www.nserc.ca/programs/ethics/english/policy.htm)defines research participants "living individuals or groups of living individuals about whom a scholar conducting research obtains(1)data through intervention or interaction with the individual or group,or(2)identifiable private information." Applying these guidelines can be helpful in determining when naturalistic observations(as,for example,noting the lengh of posting or language used in public Usenet groups)become personal interventions.Is the researcher has no interaction or intervention with the participant and if there is no disclosure of private information,then it is generally not necessary to obtain informed consent from the particlioants . Unfortunately,ethical issues are sometimes very complicated in Net-based research.making it unclear how to apply existing consent guidelines. In these cases , judgment calls must be made to defend choices that require or dispense with requirements for consent.The American Psychology association ethical code for researchers(Draft6.1 at http://www.apa.org/ethics/)notes that "before determining that planned research(such as research involving only znonymous questionnaires,naturalistic observations,or certain kinds of archival research)does not require the informed consent of research participants,psychologists consider applicable regulations and institutional review board requirements,and they consult with colleagues as appropriate." In keeping with this guideline,it is advisable that the e-researcher consult with colleagues and institutional review boards prior to dispensing with consent.

\begin{flushleft}
  \textbf{\!\!\!\!\!\!\!\!\!\!\!\!\!\!\!\!\!\!\!\!REDUCING THE POTENTIAL TO HARM}
\end{flushleft}

\!\!\!\!\!\!\!\!\!\!The second core value that underlies e-research is to insure that the e-researcher avoids,through the research process,possible harm to research participants or nonparticipants who are affected by the researcher's activities.The most common form of harm comes from inadvertent or purposeful exposure of the participants in ways that are perceived by those involved as damaging or hurtful.Examples of harm may include not only physical injuries but also loss of privileges (an inability to participate in an activity) ,inconvenience (i.e., wasted time , frustration , boredom) , psychological injuries ( insults , loos of self-esteem , embarrassment) , economic losses ( job , entrance into









\end{document}
