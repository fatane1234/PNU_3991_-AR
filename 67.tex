\documentclass[10pt,a4paper]{book}

\usepackage{graphicx}
% Requires \usepackage{graphicx}

\begin{document}
\begin{flushright}
  ETHICS AND THE E-RESEARCHER \quad \textbf{67}
\end{flushright}

\!\!\!\!\!\!\!\!\!\!(2000b)maintains there is a counter argument.While attracting vulnerable participants is an ever-present possibility,the Internet also has the ability to access participants who might otherwise be unable to particioate or who traditionally may not have been able to have a voice in research projects.For a variety of reasons(e.g.,geographic,disabilities,situational)researchers are sometimes not able to access specific people or populations.In certain circumstances,Net-based research can provide greater inclusivity by accessing these population.

Some researchers who wish to obtain consent have creatively used the forms feature on Web pages to obtain information.Figure5.1 is an example os an online consent form by Nora Boekhout(http://www.teacherwebshelf.com/) at simon Fraser University(for consent from see http:// modena.intergate.ca/personal/ boekhout/technologyincurriculum/ethicsforms.html). While the information on the form is standard for Simon Fraser University,notice that this pnline consent form has a section for a witness whereby the e-researcher may follow-up and verify the participant through the witness information.While there are no guarantees that the witness is credible(or even exists for that matter) this does provide another way to authenticate the research participants.Also note the active nature of required participation as contrasted with the way most of us casually click past license information (without reading) when first using new software packages purchased or downloaded from the NET.

\begin{figure}[h]
  \centering
  % Requires \usepackage{graphicx}
  \includegraphics[width=12cm]{5.1}\\
\end{figure}
\footnotesize\textbf{\!\!\!\!\!\!\!FIGURE 5.1}\quad Example of an Online Consent Form(Developed by Nora Bockhout.)

\end{document}
